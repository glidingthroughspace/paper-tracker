%% spellcheck-off
% -- Eigene Farben für den Quellcode
\definecolor{JavaLila}{rgb}{0.4,0.1,0.4}
\definecolor{JavaGruen}{rgb}{0.3,0.5,0.4}
\definecolor{JavaBlau}{rgb}{0.0,0.0,1.0}
\definecolor{ABAPKeywordsBlue}{HTML}{6000ff}
\definecolor{ABAPCommentGrey}{HTML}{808080}
\definecolor{ABAPStringGreen}{HTML}{4da619}
\definecolor{PyKeywordsBlue}{HTML}{0000AC}
\definecolor{PyCommentGrey}{HTML}{808080}
\definecolor{PyStringGreen}{HTML}{008080}
\definecolor{light-gray}{gray}{0.95}

% -- Default Listing-Styles

\lstset{
	% Das Paket "listings" kann kein UTF-8. Deswegen werden hier
	% die häufigsten Zeichen definiert (ä,ö,ü,...)
	literate=%
		{ä}{{\"a}}1 {ë}{{\"e}}1 {ö}{{\"o}}1 {ü}{{\"u}}1
		{Ä}{{\"A}}1 {Ë}{{\"E}}1 {Ö}{{\"O}}1 {Ü}{{\"U}}1
		{ß}{{\ss}}1 {ø}{{\o}}1  {€}{{\euro}}1,
	breaklines=true,        % Breche lange Zeilen um
	breakatwhitespace=true, % Wenn möglich, bei Leerzeichen umbrechen
	% Symbol für Zeilenumbruch einfügen
	prebreak=\raisebox{0ex}[0ex][0ex]{\ensuremath{\rhookswarrow}},
	postbreak=\raisebox{0ex}[0ex][0ex]{\ensuremath{\rcurvearrowse\space}},
	tabsize=2,                                 % Setze die Breite eines Tabs
	basicstyle=\ttfamily\small,                % Grundsätzlicher Schriftstyle
	columns=fixed,                             % Besseres Schriftbild
	numbers=left,                              % Nummerierung der Zeilen
	% frame=single,                            % Umrandung des Codes
	showstringspaces=false,                    % Keine Leerzeichen hervorheben
	keywordstyle=\color{blue},
	ndkeywordstyle=\bfseries\color{darkgray},
	identifierstyle=\color{black},
	commentstyle=\itshape\color{JavaGruen},   % Kommentare in eigener Farbe
	stringstyle=\color{JavaBlau},             % Strings in eigener Farbe,
	captionpos=b,                             % Bild*unter*schrift
	xleftmargin=5.0ex,
	backgroundcolor=\color{light-gray}
}
