%% spellcheck-off
\usepackage[ngerman]{babel}  % Neue deutsche Rechtschreibung
\usepackage[utf8]{luainputenc}
\usepackage[t1]{sourcesanspro}
\usepackage[T1]{fontenc}     % Ausgabe von westeuropäischen Zeichen (auch Umlaute)
\usepackage{graphicx}        % Einbinden von Grafiken erlauben
\usepackage{wrapfig}         % Grafiken fließend im Text
\usepackage{setspace}        % Zeilenabstand \singlespacing, \onehalfspaceing, \doublespacing
\usepackage[left=2.5cm, right=2.5cm, top=2.5cm, bottom=2.5cm,	includeheadfoot]{geometry}
\usepackage{scrlayer-scrpage}    % Gestaltung von Fuß- und Kopfzeilen
\usepackage{titletoc}            % Anpassungen am Inhaltsverzeichnis
\contentsmargin{0.725cm}         % Abstand im Inhaltsverzeichnis zw. Punkt und Seitenzahl
\usepackage[hidelinks, breaklinks=true]{hyperref}
\urlstyle{same}                  % Aktuelle Schrift auch für URLs
\usepackage{rotating} % Um figuren mit sidewaysfigure zu rotieren

% ---- Abstand verkleinern von der Überschrift
\renewcommand*{\chapterheadstartvskip}{\vspace*{.5\baselineskip}}

% ---- Für das Quellenverzeichnis
\usepackage[
	backend = biber,                % Verweis auf biber
	language = auto,
	style = numeric,                % Nummerierung der Quellen mit Zahlen
	sorting = none,                 % none = Sortierung nach der Erscheinung im Dokument
	sortcites = true,               % Sortiert die Quellen innerhalb eines cite-Befehls
	block = space,                  % Extra Leerzeichen zwischen Blocks
	hyperref = true,                % Links sind klickbar auch in der Quelle
	backref = true,                % Referenz, auf den Text an die zitierte Stelle
	bibencoding = auto,
	giveninits = true,              % Vornamen werden abgekürzt
	doi=false,                      % DOI nicht anzeigen
	isbn=false,                     % ISBN nicht anzeigen
  alldates=short                  % Datum immer als DD.MM.YYYY anzeigen
]{biblatex}
\addbibresource{literatur.bib}
\setcounter{biburlnumpenalty}{3000}     % Umbruchgrenze für Zahlen
\setcounter{biburlucpenalty}{6000}      % Umbruchgrenze für Großbuchstaben
\setcounter{biburllcpenalty}{9000}      % Umbruchgrenze für Kleinbuchstaben
\DeclareNameAlias{default}{family-given}  % Nachname vor dem Vornamen
\AtBeginBibliography{\renewcommand{\multinamedelim}{\addslash\space
}\renewcommand{\finalnamedelim}{\multinamedelim}}  % Semikolon zwischen den Autorennamen

\usepackage[babel]{csquotes}         % Anführungszeichen + Zitate

% ---- Für Mathevorlage
\usepackage{amsmath}    % Erweiterung vom Mathe-Satz
\usepackage{amssymb}    % Lädt amsfonts und weitere Symbole
\usepackage{MnSymbol}   % Für Symbole, die in amssymb nicht enthalten sind.


% ---- Für Quellcodevorlage
\usepackage{scrhack}                    % Hack zur Verw. von listings in KOMA-Script
\usepackage{listings}                   % Darstellung von Quellcode
\usepackage[dvipsnames]{xcolor}         % Einfache Verwendung von Farben
%% spellcheck-off
% -- Eigene Farben für den Quellcode
\definecolor{JavaLila}{rgb}{0.4,0.1,0.4}
\definecolor{JavaGruen}{rgb}{0.3,0.5,0.4}
\definecolor{JavaBlau}{rgb}{0.0,0.0,1.0}
\definecolor{ABAPKeywordsBlue}{HTML}{6000ff}
\definecolor{ABAPCommentGrey}{HTML}{808080}
\definecolor{ABAPStringGreen}{HTML}{4da619}
\definecolor{PyKeywordsBlue}{HTML}{0000AC}
\definecolor{PyCommentGrey}{HTML}{808080}
\definecolor{PyStringGreen}{HTML}{008080}

% -- Default Listing-Styles

\lstset{
	% Das Paket "listings" kann kein UTF-8. Deswegen werden hier 
	% die häufigsten Zeichen definiert (ä,ö,ü,...)
	literate=%
		{ä}{{\"a}}1 {ë}{{\"e}}1 {ö}{{\"o}}1 {ü}{{\"u}}1
		{Ä}{{\"A}}1 {Ë}{{\"E}}1 {Ö}{{\"O}}1 {Ü}{{\"U}}1
		{ß}{{\ss}}1 {ø}{{\o}}1  {€}{{\euro}}1,
	breaklines=true,        % Breche lange Zeilen um 
	breakatwhitespace=true, % Wenn möglich, bei Leerzeichen umbrechen
	% Symbol für Zeilenumbruch einfügen
	prebreak=\raisebox{0ex}[0ex][0ex]{\ensuremath{\rhookswarrow}},
	postbreak=\raisebox{0ex}[0ex][0ex]{\ensuremath{\rcurvearrowse\space}},
	tabsize=4,                                 % Setze die Breite eines Tabs
	basicstyle=\ttfamily\small,                % Grundsätzlicher Schriftstyle
	columns=fixed,                             % Besseres Schriftbild
	numbers=left,                              % Nummerierung der Zeilen
	%frame=single,                             % Umrandung des Codes
	showstringspaces=false,                    % Keine Leerzeichen hervorheben
	keywordstyle=\color{blue},
	ndkeywordstyle=\bfseries\color{darkgray},
	identifierstyle=\color{black},
	commentstyle=\itshape\color{JavaGruen},   % Kommentare in eigener Farbe
	stringstyle=\color{JavaBlau},             % Strings in eigener Farbe,
	captionpos=b,                             % Bild*unter*schrift
	xleftmargin=5.0ex
}  % Weitere Details sind ausgelagert

% ---- Tabellen
\usepackage{booktabs}  % Für schönere Tabellen. Enthält neue Befehle wie \midrule
\usepackage{tabularx}
%\usepackage{multirow}  % Mehrzeilige Tabellen
%\usepackage{siunitx}   % Für SI Einheiten und das Ausrichten Nachkommastellen
%\sisetup{loctolang=DE:ngerman, decimalsymbol=comma} % Damit ein Komma und kein Punkt verwendet wird.

% ---- Für Definitionsboxen in der Einleitung
%\usepackage{amsthm}                     % Liefert die Grundlagen für Theoreme
%\usepackage[framemethod=tikz]{mdframed} % Boxen für die Umrandung
%% ---- Definition für Highlight Boxen

% ---- Grundsätzliche Definition zum Style
\newtheoremstyle{defi}
  {\topsep}         % Abstand oben
  {\topsep}         % Abstand unten
  {\normalfont}     % Schrift des Bodys
  {0pt}             % Einschub der ersten Zeile
  {\bfseries}       % Darstellung von der Schrift in der Überschrift
  {:}               % Trennzeichen zwischen Überschrift und Body
  {.5em}            % Abstand nach dem Trennzeichen zum Body Text
  {\thmname{#3}}    % Name in eckigen Klammern
\theoremstyle{defi}

% ------ Definition zum Strich vor eines Texts
\newmdtheoremenv[
  hidealllines = true,       % Rahmen komplett ausblenden
  leftline = true,           % Linie links einschalten
  innertopmargin = 0pt,      % Abstand oben
  innerbottommargin = 4pt,   % Abstand unten
  innerrightmargin = 0pt,    % Abstand rechts
  linewidth = 3pt,           % Linienbreite
  linecolor = gray!40,       % Linienfarbe
]{defStrich}{Definition}     % Name der des formats "defStrich"

% ------ Definition zum Eck-Kasten um einen Text
\newmdtheoremenv[
  hidealllines = true,
  innertopmargin = 6pt,
  linecolor = gray!40,
  singleextra={              % Eck-Markierungen für die Definition
    \draw[line width=3pt,gray!50,line cap=rect] (O|-P) -- +(1cm,0pt);
    \draw[line width=3pt,gray!50,line cap=rect] (O|-P) -- +(0pt,-1cm);
    \draw[line width=3pt,gray!50,line cap=rect] (O-|P) -- +(-1cm,0pt);
    \draw[line width=3pt,gray!50,line cap=rect] (O-|P) -- +(0pt,1cm);
  }
]{defEckKasten}{Definition}  % Name der des formats "defEckKasten"  % Weitere Details sind ausgelagert

\usepackage[section]{placeins} % FloatBarrier
\usepackage{enumitem}

% Todos
\usepackage{xargs}
\usepackage[colorinlistoftodos,prependcaption,textsize=tiny]{todonotes}
\newcommandx{\NOTE}[2][1=]{\todo[linecolor=OliveGreen,backgroundcolor=OliveGreen!25,bordercolor=OliveGreen,#1]{#2}}
\newcommandx{\TODO}[2][1=]{\todo[linecolor=blue,backgroundcolor=blue!25,bordercolor=blue,#1]{#2}}
\newcommandx{\FIXME}[2][1=]{\todo[linecolor=Plum,backgroundcolor=Plum!25,bordercolor=Plum,#1]{#2}}

% Glossar, Akronyme
\usepackage[acronym,toc,automake,nopostdot,nogroupskip]{glossaries}
\makeglossaries

\def\arraystretch{1.5}
\usepackage{lscape}
\interfootnotelinepenalty=10000

\usepackage{glossary-longragged}

\renewcommand*{\glsentryfmt}{%
	\glsgenentryfmt
	\ifglsused{\glslabel}{}{\raisebox{.5em}{\tiny{\textbf{Glossar}}}}%
}

\newglossaryentry{GPS}{
	name={GPS},
	description={Das Global Positioning System (GPS) ist ein von den Vereinigten Staaten von Amerika
	entwickeltes globales Navigationssystem, welches im zivilen, kommerziellen, wissenschaftlichen und
	militärischen Bereich Anwendung findet, um die Position von GPS-Empfängern zu bestimmen.
	(vgl. \cite{GPS2008})
	}
}

\printglossary[title=Glossar, toctitle=Glossar]
