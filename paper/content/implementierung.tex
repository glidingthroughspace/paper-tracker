\chapter{Implementierung}

\section{Backend-Server}

\subsection{Verwendete Technologien}

Die grundlegend verwendete Technologie für den Backend-Server ist die Programmiersprache \enquote{Go}.
Go wurde ausgewählt, da es eine einfach zu verstehende Programmiersprache ist, die optimiert für Server-Anwendungen ist (vgl. \cite{Weigend2019}).
Zudem sind alle am Projekt beteiligten Personen schon mit der Sprache vertraut.

Go stellt eine gute Standardbibliothek zur Verfügung, die an vielen Stellen verwendet wird.
Trotzdem werden zusätzlich einige externe Bibliotheken verwendet.
Diese dienen hauptsächlich dazu, die externen Schnittstellen (\gls{HTTP} und \gls{CoAP}) mit den dazugehörigen Datenformaten zur Verfügung zu stellen.
Auch werden Bibliotheken für ein einfacheres Testing verwendet.

Im Folgenden sind die verwendeten Bibliotheken aufgelistet:
\begin{description}
	\item[logrus (https://github.com/sirupsen/logrus)] \hfill \\
		\enquote{logrus} ist strukturiertes Logging-Tool, dass für jegliche Logs innerhalb des Servers verwendet wird. 
	\item[gorm (https://github.com/jinzhu/gorm)] \hfill \\
		\enquote{gorm} ist ein \gls{ORM} Framework, dass die Datenbankanbindung für den Server zur Verfügung stellt.
	\item[Gin (https://github.com/gin-gonic/gin)] \hfill \\
		\enquote{Gin} ist ein HTTP Framework, dass Routing, Parameter-Matching, etc. übernimmt. Zudem kann es standardmäßig \gls{JSON} verstehen.  
	\item[go-coap (https://github.com/go-ocf/go-coap)] \hfill \\
		\enquote{go-coap} ist ein \gls{CoAP} Framework, dass hauptsächlich das Routing übernimmt. Einige weitere Aufgaben, die z.B. \enquote{Gin} übernimmt, müssen selbst programmiert werden.
	\item[go-codec (https://github.com/ugorji/go)] \hfill \\
		\enquote{go-codec} stellt die Codierung für \gls{CBOR} zur Verfügung.
	\item[ginkgo (https://github.com/onsi/ginkgo)] \hfill \\
		Mit \enquote{ginkgo} können Tests in \gls{BDD} Technik geschrieben werden.
	\item[gomega (https://github.com/onsi/gomega)] \hfill \\
		\enquote{gomega} ist eine Matching Library, die zu \enquote{ginkgo} gehört. Sie stellt Vergleiche zwischen erwarteten Werten und tatsächlichen Werten zur Verfügung.
	\item[GoMock (https://github.com/golang/mock)] \hfill \\
		\enquote{GoMock} ist ein Mocking Framework, dass für verschiedene Tests verwendet wird.
\end{description}

Als Datenbank wird in Kombination mit \enquote{gorm} eine \enquote{SQLite} verwendet.
Der große Vorteil ist, dass eine SQLite Datenbank eine einfache Datei ist und kein externer Server aufgesetzt werden muss.
Dies vereinfacht vor allem die Entwicklung.
Zum Beispiel ist ein einfaches Zurücksetzten der Daten durch ein Löschen der Datei möglich.

In einem späteren Einsatz des Paper-Trackers kann dank der \enquote{gorm} Bibliothek auch andere Datenbanken verwendet werden, die eher für einen dauerhaften und sichereren Einsatz vorgesehen sind.
Möglich sind dafür \enquote{MySQL}, \enquote{PostgreSQL}, \enquote{Microsoft SQL Server} und wie erwähnt SQLite.
Im Code des Servers muss hierfür lediglich eine erweiterte Konfiguration ermöglicht werden. (vgl. \cite{Jinzhu2020})

\subsection{REST-Schnittstelle}

\subsection{COAP-Schnittstelle}

\subsection{Analyse-Algorithmus}
% Wie wurde das Tracking implementiert -> Überschrift nicht ideal


\section{Hardware und Firmware}

Im Folgenden wird erläutert, wie der Tracker hinsichtlich der verwendeten Hardware und der
entsprechenden Firmware implementiert wurde.

\subsection{Verwendete Technologien}

Als Mikrokontrollerplattform wurde die sogenannte \enquote{TinyPICO}-Plattform gewählt. Nach eigener
Aussage handelt es sich dabei um die kleinste vorgefertigte Entwicklungsplattform basierend auf
Mikrocontrollern vom Typ ESP32. Dieser Aspekt ist für die Erfüllung von \ref{nf:klein} wichtig.
Dieser Mikrocontroller wurde gewählt, da er alle benötigten Funktionen wie beispielsweise \gls{WLAN}
bereits über ein \gls{API} bereitstellt.

Die TinyPICO-Entwicklungsplattform stellt neben dem Mikrocontroller weitere Hardware bereit, so sind
unter anderem
die Stromversorgung über \gls{USB} sowie Lithium-Polymer-\gls{Akku}, ein Übersetzer von \gls{USB}
zu \gls{UART} zur Programmierung des Mikrocontroller und eine Vollspektrum-\gls{LED} verbaut.
Ein weiterer Vorteil ist das bereits integrierte Ladesystem, welches einen angeschlossen
Lithium-Polymer-\gls{Akku} automatisch auflädt, sobald der Tracker über \gls{USB} mit einer
Stromquelle verbunden wird. Außerdem können über das Ladesystem der aktuelle Ladezustand und die
Spannung des \gls{Akku} abgefragt werden, was für die Erfüllung von \ref{fa:benachrichtigung}
unabdingbar ist.

Die Firmware wurde in der Programmiersprache C++ auf Basis des Arduino-Frameworks erstellt. Dieses
Framework wurde gewählt, da es auf vielen Plattformen, wie auch dem ESP32 lauffähig ist, weit
verbreitet ist und somit viele Bibliotheken für das Framework existieren.


\section{App}

Im diesem Kapitel wird die technische Umsetzung der App für Mobilgeräte beschrieben.

\subsection{Verwendete Technologien}

Zur Programmierung der App wurden die Programmiersprache Dart und das Framework Flutter verwendet.
Ein großer Vorteil dieser Kombination ist, dass Anwendungen sowohl für Android, als auch für iOS
kompiliert werden können, ohne den Quellcode anpassen zu müssen. Auch eine Laufzeitumgebung für
Desktopcomputer unter Windows, macOS und GNU/Linux befindet sich aktuell im Teststadium, sodass die
Paper-Tracker App gegebenenfalles auch auf Arbeitsplatzrechnern verwendet werden kann.

Auch für die Entwicklung der App wurden einige Bibliotheken verwendet, die die Entwicklung vereinfachen.
Vor allem werden zum Beispiel Bibliotheken verwendet, um die Kommunikation mit dem Backend-Server zu ermöglichen.
Die verwendeten Bibliotheken sind im Folgenden aufgelistet:
\begin{description}
	\item[http (https://pub.dev/packages/http)] \hfill \\
		\enquote{http} ist eine Bilbiothek, die es ermöglicht asynchron \gls{HTTP}-Anfragen durchzuführen. Die Bibliothek wird dafür verwendet, um mit dem Backend-Server zu kommunizieren.
	\item[json\_annotation (https://pub.dev/packages/json\_annotation)] \hfill \\
		Mit \enquote{json\_annotation} kann über Annotationen im Programmcode eine Serialisierung zu JSON und Deserialisierung von JSON für Datenklassen generiert werden.
	\item[shared\_preferences (https://pub.dev/packages/shared\_preferences)] \hfill \\
		Durch die \enquote{shared\_preferences} Bibliothek können einfache Konfigurationen als Schlüssel-Wert-Paare abgespeichert werden. Dies wird für die \gls{URL} des Backend-Servers verwendet.
	\item[material\_design\_icons (https://pub.dev/packages/material\_design\_icons\_flutter)] \hfill \\
		\enquote{material\_design\_icons} stellt zusätzliche Material\footnote{Eine Designsprache von Google, die in fast allen Google-Produkten eingesetzt wird und auf Minimalismus setzt. \cite{Google}}-Icons zur Verfügung. Dies ist notwendig, da die standardmäßig verfügbaren Icons in Flutter nicht ausreichend sind.
	\item[fluttertoast (https://pub.dev/packages/fluttertoast)] \hfill \\
		Mit \enquote{fluttertoast} können kleine Pop-ups dargestellt werden, die dem Nutzer der App einfach und schnell Feedback zu einer Aktion geben können.
\end{description}

\section{Bekannte Probleme}
% Erstmal ein paar Beispiele
\subsection{Ungenauigkeit des Tracking}

\subsection{UI-Details in der App}
