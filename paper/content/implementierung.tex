\chapter{Implementierung und Validierung}

\section{Backend-Server}

\subsection{Verwendete Technologien}

\subsection{Architektur}

\subsection{Analyse-Algorithmus}
% Wie wurde das Tracking implementiert -> Überschrift nicht ideal

\subsection{REST-Schnittstelle}

\subsection{COAP-Schnittstelle}


\section{Hardware und Firmware}

Im Folgenden wird erläutert, wie der Tracker hinsichtlich der verwendeten Hardware und der
entsprechenden Firmware implementiert wurde.

\subsection{Verwendete Technologien}

Als Mikrokontrollerplattform wurde die sogenannte \enquote{TinyPICO}-Plattform gewählt. Nach eigener
Aussage handelt es sich dabei um die kleinste vorgefertigte Entwicklungsplattform basierend auf
Mikrocontrollern vom Typ ESP32. Dieser Aspekt ist für die Erfüllung von \ref{nf:klein} wichtig.
Dieser Mikrocontroller wurde gewählt, da er alle benötigten Funktionen wie beispielsweise \gls{WLAN}
bereits über ein \gls{API} bereitstellt.

Die TinyPICO-Entwicklungsplattform stellt neben dem Mikrocontroller weitere Hardware bereit, so sind
unter anderem
die Stromversorgung über \gls{USB} sowie Lithium-Polymer-\gls{Akku}, ein Übersetzer von \gls{USB}
zu \gls{UART} zur Programmierung des Mikrocontroller und eine Vollspektrum-\gls{LED} verbaut.
Ein weiterer Vorteil ist das bereits integrierte Ladesystem, welches einen angeschlossen
Lithium-Polymer-\gls{Akku} automatisch auflädt, sobald der Tracker über \gls{USB} mit einer
Stromquelle verbunden wird. Außerdem können über das Ladesystem der aktuelle Ladezustand und die
Spannung des \gls{Akku} abgefragt werden, was für die Erfüllung von \ref{fa:benachrichtigung}
unabdingbar ist.

Die Firmware wurde in der Programmiersprache C++ auf Basis des Arduino-Frameworks erstellt. Dieses
Framework wurde gewählt, da es auf vielen Plattformen, wie auch dem ESP32 lauffähig ist, weit
verbreitet ist und somit viele Bibliotheken für das Framework existieren.


\section{App}

Im diesem Kapitel wird die technische Umsetzung der App für Mobilgeräte beschrieben.

\subsection{Verwendete Technologien}

Zur Programmierung der App wurden die Programmiersprache Dart und das Framework Flutter verwendet.
Ein großer Vorteil dieser Kombination ist, dass Anwendungen sowohl für Android, als auch für iOS
kompiliert werden können, ohne den Quellcode anpassen zu müssen. Auch eine Laufzeitumgebung für
Desktopcomputer unter Windows, macOS und GNU/Linux befindet sich aktuell im Teststadium, sodass die
Paper-Tracker App gegebenenfalles auch auf Arbeitsplatzrechnern verwendet werden kann.

\section{Bekannte Probleme}
% Erstmal ein paar Beispiele
\subsection{Ungenauigkeit des Tracking}

\subsection{UI-Details in der App}
