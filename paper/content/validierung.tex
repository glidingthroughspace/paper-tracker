\chapter{Validierung}
Mit Hilfe der Validierung soll festgestellt werden, ob die gestellten Anforderungen erreicht wurden.
Besonders interessant sind im Falle des Paper-Trackers zum einen die Genauigkeit, mit der ein Tracker lokalisiert werden
kann und zum anderen, wie gut das Gesamtsystem miteinander funktioniert und bedienbar ist.
Auch das Setup eines neuen Paper-Tracker-Systems soll dabei mit berücksichtigt werden.

\section{Genauigkeitsmessung des Tracking}
\subsection{Beschreibung}
Bei der Validierung der Genauigkeit des Trackings soll in mehreren Versuchen ermittelt werden, wie genau die Position
eines Dokuments mit Hilfe der Technik des Paper-Trackers bestimmt werden kann.
Das zu erreichende Ziel, um die gestellten Anforderungen zu erzielen, ist eine raumgenaue Ortung (\ref*{fa:tracking}).

\subsection{Durchführung}
Um die Genauigkeit des Systems zu messen, wird eine Reihe an Versuchen durchgeführt, die alle einem gemeinsamen Muster folgen:
\begin{enumerate}
	\item Zurücksetzen des kompletten Systems
	\item Hinzufügen des Trackers
	\item Hinzufügen der in diesem Versuch verwendeten Räume
	\item Einlernen der Räume in beliebiger Reihenfolge
	\item Räume in gegebener Tracking Reihenfolge abgehen
	\begin{enumerate}
		\item Für einige Sekunden im Raum verbleiben
		\item Im Abstand von einigen Sekunden an verschiedenen Stellen im Raum die Position des Trackers bestimmen
	\end{enumerate}
\end{enumerate}
Ein Versuch gilt als erfolgreich, wenn alle Räume in gegebener Reihenfolge richtig erkannt werden bzw. auch in einem
Negativ-Test nicht erkannt werden.

Die auszuführenden Versuche sind sortiert von \enquote{einfachen} zu \enquote{schwierigen} Situationen.
Als einfach zählt zum Beispiel ein Versuch mit nur einem Raum und zu schwierig, wenn als Raum nur eine
Raumhälfte verwendet wird.
Folgendes sind die auszuführenden Versuche:
\begin{enumerate}
	\item Ein Raum wird eingelernt und in diesem getrackt
	\item Ein Raum wird eingelernt und in einem entfernten Raum getrackt
	\item Ein Raum wird eingelernt und in einem angrenzenden Raum getrackt
	\item Zwei entfernte Räume werden eingelernt und in selbiger Reihenfolge getrackt
	\item Zwei angrenzende Räume werden eingelernt und in selbiger Reihenfolge getrackt
	\item Ein Raum wird in zwei Raumhälften geteilt, welche als Räume getrackt werden und in selbiger Reihenfolge getrackt
\end{enumerate}

Zu den Ergebnissen der Versuche selbst muss auch das Umfeld der Versuche angegeben werden.
Dies ist das Gebäude, in dem der Versuch stattgefunden hat, eine ungefähre Raumgröße, die Entfernung der Räume zueinander
und die Struktur des für das Tracking verwendeten \gls{WLAN}-Netzes.

\subsection{Ergebnisse}
Aufgrund der aktuellen Coronavirus-Lage (\cite{bw2020}, \cite{dhbwka2020}) während dem Bearbeitungszeitraum dieser Arbeit,
ist es leider nicht möglich diese Validierung entsprechend durchzuführen.
Grund dafür ist ein fehlendes passendes Gebäude mit Zimmern, die teilweise entfernt voneinander liegen, und einem gut
genug ausgebauten \gls{WLAN} Netzwerk. 

Trotzdem können eingeschränkt zu den ersten beiden Tests Aussagen gemacht werden, da diese vor der Coronavirus-Lage
getestet wurden.
Hierbei muss angemerkt werden, dass die oben beschrieben Versuchsabläufe nicht angewendet wurden sondern frei getestet wurde.
Das Umfeld des Versuches war die \gls{DHBW} Karlsruhe.
In dieser wurden zwei Vorlesungsräume in einem Gang verwendet.
Die Entfernung zwischen den zwei Räumen, die an unterschiedlichen Seiten des Ganges abgehen, betrug ungefähr fünf Meter.
Als \gls{WLAN}-Netz zum Tracken wurden alle verfügbaren Netze verwendet (\enquote{eduroam}, \enquote{DHBW-KA} und weitere kleinere Netze).

Unter diesen Umständen konnten beide Tests erfolgreich abgeschlossen werden.
Dies bestätigt zumindest teilweise die Anforderung für eine raumgenaue Ortung.
Für eine komplette Bestätigung der Erfüllung, müssten alle Versuche durchgeführt werden.

\section{Einfaches Setup und Nutzung des Gesamtsystems}
\subsection{Beschreibung}
In dieser Validierung des Projektes soll bewertet werden, wie einfach ein potentieller Nutzer/Betreiber
des Paper-Trackers ein neues System aufsetzen und verwenden kann.
Um dies zu bestimmen, soll eine oder mehrere Testpersonen alleine mit den Informationen aus dem GitHub-Repository, das
den kompletten Code und die Dokumentation beinhaltet, das System aufsetzen und in Betrieb nehmen.
Im Nachhinein wird den Testpersonen einige Fragen zum Prozess gestellt und sie können selbst freies Feedback geben.

\subsection{Durchführung}
Im Folgenden wird beschrieben, wie der Testverlauf für eine Testperson aussieht.
Auch werden die zu erwartenden Schritte, die die Testperson ausführen soll, angegeben.
Die Testpersonen sollten möglichst mit Informatik versiert sein und zuvor schon einmal einen Server mit einer
Applikation in Betrieb genommen haben.
Als Plattform für die App soll, wenn möglich, Android verwendet werden, da dafür bereits fertige Installationsdateien
zur Verfügung stehen.
Für Apple-Geräte ist dies leider nicht möglich.

Zum Start des Tests bekommt die Testperson einige Informationen, was der Paper-Tracker ist, wofür er bestimmt ist
und die Aufgabe, das System in Betrieb zu nehmen. 
Weiter bekommt sie als Ressource, wie oben beschrieben, das GitHub-Repository des Paper-Trackers.
Dieses bietet im Detail die folgenden Komponenten:
\begin{itemize}
	\item Source-Code zu der Firmware, dem Flasher, der App und dem Server
	\item Implizit im Server-Code die App und den ausführbaren Flasher
	\item Ein \enquote{Dockerfile} für den Server
	\item Eine \enquote{README}-Datei mit Informationen zum Setup des Servers
	\item Dieses Paper mit weiterführenden Informationen zum kompletten Projekt
\end{itemize}

Mit Hilfe dieser Komponenten werden folgende Schritte von der Testperson erwartet:
\begin{enumerate}
	\item Setup des Servers
	\begin{enumerate}
		\item Lesen der \enquote{README}-Datei
		\item Anpassen der Konfigurationsdatei mit geeigneten Werten
		\item Erstellen des Docker-Images und Start von diesem auf einem Server \hfill \\
			\hspace*{20mm} oder \hfill \\
			Lokal den Code kompilieren, kopieren der ausführbaren Datei auf den Server, Starten der Datei
	\end{enumerate}
	\item Aufrufen der Download Seite des Servers
	\item Download der Android App und Installation
	\item Durcharbeiten des Tutorials innerhalb der App inklusive des Flashen eines Trackers
\end{enumerate}

\subsection{Ergebnisse}


\subsection{Gesamtergebniss der Validierung}