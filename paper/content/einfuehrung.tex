\chapter{Einführung}

An der \gls{DHBW} in Karlsruhe basieren, wie auch noch in einigen anderen Organisationen, viele Prozesse auf Papier.
Dies bedeutet, dass beispielsweise für die Genehmigung eines Einkaufes ein oder mehrere Dokumente von unterschiedlichen Personen unterzeichnet werden müssen.
Dafür wird das Dokument durch die einzelnen Büros getragen, bis letztendlich der Antragssteller es wieder bekommt.

Auch für simple Prozesse kann dies einige Zeit in Anspruch nehmen.
Während dieser ganzen Zeit hat der Antragsteller leider keine Einsicht, bei welcher Person sich die Papiere befinden.
Dadurch kann er auch nicht versuchen, den Prozess zu beschleunigen, ohne Schritt für Schritt bei allen Personen des Prozesses nachzufragen.
Zusätzlich kann nach Abschluss des Prozesses nicht nachvollzogen werden, wo der Flaschenhals des Prozesses liegt.

Um dieses Problem anzugehen, wird der \enquote{Paper-Tracker} entwickelt.
Dieser soll eine Transparenz in alle auf Papier basierten Prozesse bringen und zusätzlich Daten liefern, um die Prozesse selbst zu optimieren.

\section{Idee des Paper-Tracker}

Der Paper-Tracker ist ein kleines Gerät basierend auf einem Mikrocontroller, welcher an den Dokumenten befestigt wird.
Dieser soll eine Lokalisierung der Papiere ermöglichen.
Auf diese Lokalisierung soll der Antragsteller eines Prozesses über eine App Zugriff bekommen.
Weiter soll auch in der App für Mobilgeräte der komplette Prozess oder Workflow selbst verfolgt werden.
Somit kann zum Beispiel auch eine Notifizierung zuständiger Personen für einen Schritt in einem Workflow erfolgen.
