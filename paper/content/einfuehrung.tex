\chapter{Einführung}

An der \gls{DHBW} in Karlsruhe basieren, wie auch noch in vielen anderen Organisationen, viele Prozesse auf Papier.
Dies bedeutet, dass beispielsweise für die Genehmigung eines Einkaufes, ein oder mehrere Blätter Papier von unterschiedlichen Personen unterzeichnet werden müssen.
Dafür wird das Blatt durch die einzelnen Büros getragen, bis letztendlich der Antragssteller es wieder bekommt.

Auch für simple Prozesse kann diese einige Zeit in Anspruch nehmen.
Während dieser ganzen Zeit hat der Antragsteller leider keine Einsicht, bei welcher Person sich die Papiere befinden.
Damit kann er auch nicht versuchen den Prozess zu beschleunigen, ohne Schritt für Schritt bei allen Personen des Prozesses nachzufragen.
Zusätzlich kann er im Nachhinein nicht herausfinden, weshalb der Prozess länger als erwartet gebraucht hat.

Um dieses Problem anzugehen, wurde der \enquote{Paper-Tracker} entwickelt.
Dieser soll eine Transparenz in alle auf Papier basierten Prozesse bringen und zusätzlich Daten liefern, um die Prozesse selbst zu optimieren.

\section{Idee des Paper-Tracker}

Die Idee des Paper-Trackers basiert darauf, einen möglichst kleinen Tracker basierend auf einem Mikrocontroller an den Papieren eines Prozesses zu befestigen.
Dieser soll eine Lokalisierung der Papiere ermöglichen.
Auf diese Lokalisierung soll der Antragsteller eines Prozesses über eine App Zugriff bekommen.
Weiter soll auch in der App der komplette Prozess oder Workflow selbst verfolgt werden.
Somit kann zum Beispiel auch eine Notifizierung zuständiger Personen für einen Schritt in einem Workflow erfolgen.
