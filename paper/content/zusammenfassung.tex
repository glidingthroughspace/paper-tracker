\chapter{Zusammenfassung}

Mit Sicht auf die Erfüllung der Anforderungen und erfolgreichen Validierung kann das Paper-Tracker Projekt als
Erfolg angesehen werden.

Hinsichtlich der Anforderungen konnte ein System implementiert werden, dass mit Hilfe eines Hardware-Trackers Dokumente
raumgenau lokalisieren kann, um damit Prozesse nachverfolgen zu können.
Die Prozesse können in einer App erstellt und deren Verlauf überwacht werden.
Alle für einen Prozess relevante Räume werden ebenfalls über die App im System angelegt und können mit Hilfe eines
Trackers eingelernt werden.
Lediglich die Zuweisung mehrerer Räume für einen Schritt in einem Prozess ist nicht als Funktionalität in der App
verfügbar sondern nur über die direkte Server-Schnittstelle.
Um Prozesse, die mit dem Paper-Tracker verfolgt wurden, zu analysieren, ist ein Excel-Export vorhanden.
Dieser bietet eine gut Datenbasis, um Verbesserungen und Optimierungen in einem Prozess identifizieren zu können.

Mit der Validierung konnte vor allem die Lokalisierung eines Tracker auf die Genauigkeit eines Raumes gezeigt werden.
Auch wurde die einfache Bedienung des Gesamtsystems über die App durch einen externen Tester überprüft und bestätigt.

Die Hardware für den Tracker selbst wurde so ausgewählt, dass dieser möglichst kompakt ist aber trotzdem genügend
Batterielaufzeit auch für längere Prozesse bietet.
Für diese ausgewählte Hardware wurde zudem ein Gehäuse entwickelt und im 3D-Drucker gedruckt, um die Hardware
einfach an einem Stapel Papier oder einem Biref befestigen zu können.

Trotz dem Erfolg des Projektes konnten einige Verbesserungen und Ergänzungen für eine mögliche Fortführung gezeigt werden.
So kann zum Beispiel im Design der App und bei der Zuverlässigkeit der Messung des Batteriestandes eine bessere Qualität erzielt werden.
Als mögliche Erweiterungen, die den Umfang des Projektes um einiges erhöhen würden, wurden ein Usermanagement und das Tracking weiterer Hardware skizziert.
