\chapter{Ausblick und Weiterentwicklungen}

Mit dem aktuellen Stand des Paper-Trackers konnten nahezu alle Anforderungen erfüllt werden. Einzig \ref*{fa:mehrere-orte} wurde nicht vollständig implementiert.
Trotzdem gibt es viele Ansätze, mit denen das Projekt weiterentwickelt werden kann.
In den folgenden Abschnitten werden einige mögliche Weiterentwicklungen skizziert und damit ein Ausblick gegeben,
wie das Projekt fortgeführt werden kann.

\section{Audiovisuelle Signale}

Mit dem aktuellen Stand kann ein Dokument, das mit dem Paper-Tracker ausgestattet ist, auf die Genauigkeit eines Raumes
lokalisiert werden.
Innerhalb des Raumes kann es aber trotzdem schwierig sein, das richtige Dokument zu finden, wenn es zum Beispiel von
weiteren Dokumenten verdeckt wird.
Eine Möglichkeit, dieses Problem zu lösen, wäre das Hinzufügen von audiovisuellen Signalen zu dem Tracker selbst.
Dies könnte zum Beispiel eine \gls{LED} in Kombination mit einem Lautsprecher, Summer oder Piepser sein.
Der Endnutzer könnte diese Signale über die App auslösen und somit das Dokument innerhalb eines Raumes lokalisieren.

\section{Erhöhte Tracking-Genauigkeit}

Um vor allem das Tracking von Dokumenten in von mehreren Personen genutzten Räumen, wie zum Beispiel einem Großraumbüro,
zu verbessern, wäre eine erhöhte Genauigkeit erforderlich.
Mit Hilfe von Verbesserungen des Algorithmus wäre solch eine Verbesserung denkbar.

\section{Nutzer und Benachrichtigungen}

In der aktuellen Version des Paper-Trackers kann jede Person, die die Adresse des Servers kennt, interne Informationen
abfragen.
Dies ist ein großes Sicherheitsproblem, welches über ein Nutzermanagement gelößt werden kann, indem sich die Benutzer
anmelden müssen.

Weiter können die Nutzer dafür verwendet werden, ihnen einzelne Schritte in einem Workflow, einzelne Räume
oder Workflow Ausführungen zuzuweisen.
Damit könnte der Nutzer in der App einsehen, welche Dokumente seine Aufmerksamkeit benötigen oder wo sich eigene Dokumente
zum aktuellen Zeitpunkt befinden.

Mit der Information, welcher Nutzer einem Schritt, Raum oder Ausführung zugeordnet ist, könnte der Server personalisiert
Benachrichtigungen verschicken.
Denkbar wären Benachrichtigungen, wenn eine eigene Ausführung einen Fortschritt gemacht hat, ein Dokument im Büro
des Nutzers ist und mehr.

\section{Tracking von Smartphones und Laptops}

Momentan wird die Tracking-Technologie nur für die Tracker selbst verwendet.
Es wäre allerdings auch möglich, mit Hilfe der gleichen Technologie auch Smartphones oder Laptops zu tracken.
In Kombination mit dem Nutzermanagement und Benachrichtigungen wäre es vorstellbar, einen Nutzer zu benachrichtigen
wenn er einen Raum betritt, in dem ein Dokument auf ihn wartet.

Da dadurch letztendlich die Nutzer selbst getrackt werden, müsste hier besonders auf den Datenschutz geachtet werden.
Es sollte nicht möglich sein, dass der Betreiber des Paper-Trackers oder der Manager dadurch Mitarbeiter nachverfolgen kann.
