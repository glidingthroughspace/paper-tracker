\chapter{Analyse} \label{chap:analyse}

In den folgenden Abschnitten werden der Ist-Zustand, aus welchem sich die Problemstellung ergibt,
der Soll-Zustand, sowie die gestellten Anforderungen beschrieben.

\section{Ist-Analyse} \label{sec:ist-analyse}

Es besteht das Problem, dass der aktuelle Status eines auf Papier basierten Prozesses nicht nachvollzogen werden kann.
Für den Antragsteller eines Prozesses ist es nicht möglich herauszufinden, bei welcher Person oder in welchem Raum sich ein Dokument
zu einem bestimmten Zeitpunkt befindet.
Zusätzlich kann, sobald der Prozess beendet ist, nicht herausgefunden werden, an welcher Stelle im Prozess zum Beispiel eine besonders
hohe Wartezeit stattgefunden hat.

\section{Soll-Analyse} \label{sec:soll-analyse}

Ziel der Entwicklung und damit dieser Studienarbeit ist das Erarbeiten einer Lösung, mit welcher die aktuelle Position eines Dokuments und der aktuelle Fortschritt des Prozesses sichtbar gemacht werden kann.
Es soll ein möglichst unauffälliges Gerät sein und eine intuitive Bedienung ermöglichen.
Über aktuelle und abgeschlossene Prozesse sollen Daten analysiert werden können, um Möglichkeiten zur Prozessoptimierungen erkennen zu können.

\section{Anforderungsanalyse} \label{sec:anforderungsanalyse}

Im Folgenden werden die gestellten Anforderungen in funktionale und nichtfunktionale Anforderungen
unterteilt.
Dabei beschreiben funktionale Anforderungen konkrete Funktionen, die die entwickelte Lösung bieten
muss, nichtfunktionale Anforderungen hingegegen zeigen Rahmenbedingungen, sowie Anforderungen an die
technische Umsetzung auf.
Desweiteren werden den Anforderungen Prioritäten zugewiesen, wobei ein Wert von 0 die höchste
Priorität, ein Wert von 3 die niedrigste Priorität beschreibt, sodass die Lösung ohne Erfüllung der
Anforderungen mit Priorität 0 nicht einsetzbar ist und Priorität 3 optionale Funktionalitäten
beschreibt, die nicht zum Betrieb notwendig sind.

\begin{enumerate}[label=\textbf{F-\arabic*}]
	\item \label{fa:tracking} \textit{P0} Der Standort von Dokumenten kann raumgenau verfolgt werden
	\item \label{fa:workflow} \textit{P0} Der aktuelle Status des mit dem Dokument verbundenen
		Workflows kann nachvollzogen werden
	\item \label{fa:neue-workflows} \textit{P2} Neue Workflows können vom Nutzer definiert werden
	\item \label{fa:neue-tracker} \textit{P1}  Neue Tracker können dem System vom Nutzer hinzugefügt werden
	\item \label{fa:raumplan} \textit{P2}  Der Raumplan kann vom Nutzer eingelernt werden
	\item \label{fa:benachrichtigung} \textit{P3} Nutzer werden vom System benachrichtigt, wenn die Akkuladung eines
	Trackers unter einen gegebenen Prozentsatz fällt oder die Batterie ausgetauscht werden muss
	\item \label{fa:optionale-schritte} \textit{P1} Einzelne Schritte von Workflows können optional sein und
	übersprungen werden
	\item \label{fa:mehrere-orte} \textit{P3}  Für einen Schritt im Workflow können mehrere Orte hinterlegt werden
	\item \label{fa:export} \textit{P1} Es soll ein Daten-Export in ein passendes Format möglich sein, das für Analysen \\
	  zu Optimierungen genutzt werden kann
	\item \label{fa:arbeitszeiten} \textit{P3} Es sollen Arbeitszeiten angegeben werden können. \\
		Der Tracker soll nur innerhalb der Arbeitzeiten aktiv sein.
	\item \label{fa:wochenende} \textit{P3} Es soll einstellbar sein, dass der Tracker am Wochenende nicht aktiv ist
\end{enumerate}


\begin{enumerate}[label=\textbf{NF-\arabic*}]
	\item \label{nf:akku} \textit{P2}  Die Laufzeit der Tracker im Batteriebetrieb beläuft sich auf mindestens eine Woche
	\item \label{nf:klein} \textit{P1}  Die Tracker sollen möglichst unscheinbar und daher klein und leicht sein
	\item \label{nf:app} \textit{P0}  Die Daten können in einer Applikation für Mobiltelefone abgefragt werden.  Dabei ist besonders die Verwendbarkeit unter dem Betriebssystem Android wichtig, die Unterstützung von iOS ist optional
	\item \label{nf:stabilität} \textit{P1}  Die Anwendung muss stabil laufen und sollte sich nicht unerwartet beenden
	\item \label{nf:skalierbarkeit} \textit{P2}  Die Lösung sollte skalierbar hinsichtlich der Anzahl der Tracker und der Größe des Tracking-Bereiches sein
	\item \label{nf:nachvollziehbarkeit} \textit{P2}  Zur optimalen Nachvollziehbarkeit von Workflows sollten deren auch nach Abschluss nicht gelöscht werden
\end{enumerate}

\section{Stand der Forschung} \label{sec:stand-der-forschung}

Positionsbestimmung in Gebäuden ist ein Problem, zu welchem es viel Forschung gibt, da bestehende
Lokalisierungsdienste wie \gls{GPS} in Gebäuden nur unzureichend nutzbar sind (vgl. \cite{Liu}). Dieser Abschnitt soll
daher eine grobe Übersicht über den Stand der Forschung bieten und bestehende Arbeit mit den
Anforderungen und der Methodik dieser Arbeit abgleichen.

\subsection{Relative Positionsmessung in Meshnetzwerken}

Eine Möglichkeit, Geräte innerhalb von Gebäuden zu lokalisieren wird in \cite{Patwari2003}
geschildert. In der hier vorgeschlagenen Methode wird ein \gls{Mesh-Netzwerk} aus Sensoren aufgebaut.
In diesem Mesh können sich die Sensoren nun untereinander Signale zusenden und dann die
Empfangsstärke oder sogenannte Time-of-Arrival messen, um die Entfernung zu Nachbarn abzuschätzen.
So kann für jeden Sensor die relative Position zu anderen Sensoren berechnet werden.
Ist die Position einiger weniger Knoten des Meshes bekannt, kann auch die absolute Position aller
anderen Sensoren angegeben werden.

Mit dieser Methode kann eine Genauigkeit von bis zu einem Meter bei der Lokalisierung erzielt
werden.

Für den hier vorgestellten Anwendungsfall ist dieser Ansatz jedoch nicht geeignet, da im gesamten
Gebäude Sensoren verteilt werden müssen, um eine lückenlose Positionierung gewährleisten zu können.

\subsection{Lokalisierung durch \glsentryshort{WLAN}-Netzwerke}

Eine weitere Möglichkeit ist, auf bestehende Infrastruktur zurückzugreifen. \Gls{WLAN} ist
heutzutage in praktisch jedem Gebäude verfügbar.
In Gebäuden, in welchen es mehrere \glspl{AP} gibt, wie es unter anderem in der \gls{DHBW} der Fall
ist, kann das \gls{WLAN}-Netz für die Lokalisierung von Geräten verwendet werden. Dies wird unter
anderem in \cite{Xiang2004}, \cite{Dong2009}, \cite{Paschalidis2009} und \cite{Liang2009} gezeigt.
In allen diesen Werken wird die Position eines Endgerätes dadurch bestimmt, dass ausgewertet wird,
von welchen \gls{WLAN}-\glspl{AP} das Endgerät ein Signal empfängt und wie stark dieses ist.

Dazu werden Mobile Geräte, wie Mobiltelefone und Laptops mit einer Software ausgestattet, welche in
regelmäßigen Abständen alle \glspl{AP} in seiner Umgebung abruft und die jeweilige Signalstärke
speichert. Die vorgestellten Lokalisierungsalgorithmen basieren darauf, dass \gls{WLAN}-Umgebungen
\enquote{eingelernt} werden und dann neue Messungen auf die gelenten gematched werden.

Über diese Verfahren können Geräte mit bis zu $95\%$ Sicherheit auf einem Meter genau lokalisiert
werden. Diese Genauigkeit ist ausreichend für den gewünschten Verwendungszweck.
