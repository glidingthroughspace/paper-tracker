\chapter{Analyse} \label{chap:analyse}

In den folgenden Abschnitten werden der Ist-Zustand, aus welchem sich die Problemstellung ergibt,
der Soll-Zustand, sowie die gestellten Anforderungen beschrieben.

\section{Ist-Analyse} \label{sec:ist-analyse}

Es gibt Probleme bei der Nachvollziehbarkeit von Workflows. Es ist undurchsichtig, was gerade
passiert. Außerdem kommen Papiere durcheinander.

\section{Soll-Analyse} \label{sec:soll-analyse}

\section{Anforderungsanalyse} \label{sec:anforderungsanalyse}

Im Folgenden werden die gestellten Anforderungen in funktionale und nichtfunktionale Anforderungen
unterteilt. 
\TODO{Evtl. noch eine tatsächliche Anforderungsanalyse durchführen}
Dabei beschreiben funktionale Anforderungen konkrete Funktionen, die die entwickelte Lösung bieten
muss, nichtfunktionale Anforderungen hingegegen zeigen Rahmenbedingungen, sowie Anforderungen an die
technische Umsetzung auf.
Desweiteren werden den Anforderungen Prioritäten zugewiesen, wobei ein Wert von 0 die höchste
Priorität, ein Wert von 3 die niedrigste Priorität beschreibt, sodass die Lösung ohne Erfüllung der
Anforderungen mit Priorität 0 nicht einsetzbar ist und Priorität 3 optionale Funktionalitäten
beschreibt, die nicht zum Betrieb notwendig sind. 
\begin{table}[htbp]
  \renewcommand{\arraystretch}{1.5}
\centering
\begin{tabularx}{0.9\textwidth}{lXc}
  \textbf{Name} & \textbf{Beschreibung} & \textbf{Priorität} \\
  \hline 
  F-1 \label{fa:tracking} & Der Standort von Dokumenten kann raumgenau verfolgt werden & 0 \\
  F-2 \label{fa:workflow} & Der aktuelle Status des mit dem Dokument verbundenen Workflows kann
  nachvollzogen werden & 0 \\
  F-3 \label{fa:neue-workflows} & Neue Workflows können vom Nutzer definiert werden & 2 \\
  F-4 \label{fa:neue-tracker} & Neue Tracker können dem System vom Nutzer hinzugefügt werden & 1 \\
  F-5 \label{fa:raumplan} & Der Raumplan kann vom Nutzer eingelernt werden & 3 \\
  \midrule
  NF-1 \label{nf:akku} & Die Laufzeit der Tracker im Batteriebetrieb beläuft sich auf mindestens
  eine Woche & 2 \\
  NF-2 \label{nf:klein} & Die Tracker sollen möglichst unscheinbar und daher klein und leicht sein
  & 1 \\
  NF-3 \label{nf:app} & Die Daten können in einer Applikation für Mobiltelefone abgefragt werden.
  Dabei ist besonders die Verwendbarkeit unter dem Betriebssystem Android wichtig, die Unterstützung
 von iOS ist optional & 0 \\
  NF-4 \label{nf:stabilität} & Die Anwendung muss stabil laufen und sollte sich nicht unerwartet
  beenden & 1 \\
  NF-5 \label{nf:skalierbarkeit} & Die Lösung sollte skalierbar hinsichtlich der Anzahl der Tracker
  und der Größe des Tracking-Bereiches sein & 2 \\
  NF-6 \label{nf:nachvollziehbarkeit} & Zur optimalen Nachvollziehbarkeit von Workflows sollten deren auch
  nach Abschluss nicht gelöscht werden & 2 \\
\end{tabularx}
\caption{Funktionale und nichtfunktionale Anforderungen}
\label{tab:anforderungen}
\end{table}
